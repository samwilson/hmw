\biohead{Constance Boyd Hall}{}

Connie Berryman (n\'{e}e Hall), eldest daughter of Aubrey and Helen Hall, was born in Roeburn on 9 August 1912.\cite{ConnieBirthNotice}

Connie married Alan Berryman (\p{Alan_Louis_Mathew_Berryman}) on Saturday 20 May 1937 in Canarvon. Their wedding was reported on the following Wednesday in the \emph{Northern Times}:\cite{ConnieWedding}

\begin{quotation}
Wedding Bells.

BERRYMAN-HALL

The marriage took place at St. George's Church, Carnarvon on Saturday May 20 at 12.30 noon, of Constance Boyd, eldest daughter of Mr. and Mrs. Aubrey Hall of Carnarvon, and Mr. Alan Berryman (of Elder, Smith and Co. Ltd. Carnarvon), son of Mrs. Berryman of Birmingham, Engand, The Ven. Archdeacon Simpson performing the wedding ceremony. The bride who was given away by her father, wore a beautiful lace over ninon frock cut on slim lines, with long train and long sleeves. The misty tulle veil which was suspended from silver halo, was the one worn by the bride's mother at her wedding, and the bride carried an early Victorian posy of pale pink roses and paw-paw flowers. The altar was decorated by Mrs. Eden with yellow chrysautheums and Mrs. Horton officiated at the organ.

After the ceremony a luncheon party was held at the home of the bride's parents which was decorated with a     profusion of gorgeous roses, the gift of Mrs. Quince. Mrs. Hall received the guests, wearing a smart frock of navy figured morocain, with hat to tone.

A beautifully decorated square wedding cake, surmounted with the Leake family vase, which had adorned both the bride's mother's and grandmother's wedding eakes and is loaned for all family weddings, held pride of place on the bridal table.

After the usual toasts' had been honored, the bride and bridegroom left by car for Wooramel station, where the honeymoon was spent, the bride travelling in a tunic frock of cream woollen matelasse with button trim mings and navy blue accessories.

Mr. and Mrs. Berryman were the recipients of many handsome gifts and cheques.
\end{quotation}
